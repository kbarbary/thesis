
We now use all available information about each candidate
(spectroscopic confirmation, host galaxy redshift, all light curve
information, as well as host galaxy luminosity and color) to classify
each of the 60 remaining candidates as image artifact, active galactic
nucleus (AGN), core-collapse SN (SN~CC), or SN~Ia.
%Because the amount of information varies greatly among candidates,
%we must consider each candidate on an individual basis, meaning that a
%selection efficiency cannot be automatically computed. Instead, we
%estimate a systematic error in the final result from possible
%mis-classifications in this step. 
%Note that although not every candidate is discussed individually here,
%all 60 candidates can be viewed in detail
%at \url{http://supernova.lbl.gov/cands/cands.html}.

%We will show that for nearly all candidates classified as AGN or image
%artifacts, we can say with certainty that they are \emph{not} SNe~Ia
%on the basis of the complete light curve and limits on host galaxy
%redshifts.  We will also show that most of the remaining candidates
%can be unambiguously classified as SNe~Ia or SNe II. 
