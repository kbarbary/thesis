
If the real distributions of SN~Ia properties differs significantly
from those assumed in our simulation, the $T(x,y)$ maps we have
derived could misrepresent the true efficiency of the survey. Above we
argued that the effect is likely to be small because the detection
efficiency is close to 100\% for most of the survey. Here we quantify
the size of the possible effect on the control time by varying the
assumed distributions.

To first order, changing the assumed distributions of $s$ or $c$ or
changing the assumed spectral time series will affect the detection
efficiency by increasing or decreasing the luminosity of the simulated
SN. To jointly capture these effects, we shift the absolute magnitude
of the simulated SNe~Ia by $^{+0.2}_{-0.2}$~mag and recalculate the
control times. To first order, this is equivalent to shifting the $s$
distribution by $\Delta s = 0.2/\alpha \sim 0.16$ or shifting the $c$
distribution by $\Delta c = 0.2/\beta \sim 0.09$. A $-0.2$~mag shift
in absolute magnitude increases the control time, decreasing the
inferred SN~Ia rate by $6\%$. A $+0.2$~mag shift decreases the control
time, increasing the SN~Ia rate by $8\%$. These effects are
sub-dominant compared to the Poisson error of $\gtrsim 30\%$ in the
number of SNe observed. (Sources of error are summarized
in \S\ref{sec:clrate_results_sys} and Table~\ref{tab:clrate_sys}.)

% EXPLANATION OF SPECTRAL TIME SERIES CHOICE
%To generate the simulated light curves in the observed bands, we use
%the \citet{hsiao07a} SN~Ia spectral time series template.
%This template is known to have higher fidelity in the rest-frame UV
%than, e.g., the {\sc salt}
%\citep{guy05a} or \citet{nugent02a} templates. An accurate UV SN
%template is particularly important here, as the observed $z_{850}$
%band corresponds to rest-frame $U$-band at $z \sim 1.4$.  

%EXPLANATION OF STRETCH DISTRIBUTION
%We keep in mind that clusters are dominated
%by passive galaxies, which preferentially host lower-stretch SNe: we
%base our stretch distribution on the first-year Supernova Legacy
%Survey (SNLS) sample of $z<0.75$ SNe in passive hosts \citet{sullivan06a}.

%EXPLANATION OF COLOR DISTRIBUTION
%For color, we follow a similar approach, but base
%our distribution on the full first-year SNLS sample \citep{astier06a}
%of $z<0.6$ SNe in all host types (the observed $c$ distribution is not
%significantly different in early-type hosts). \citep[e.g.,][]{sullivan10a}
%We have cut the sample at $z<0.6$ to minimize Malmquist bias against
%redder SNe.  The $c$ distribution of these SNe is reproduced in the
%left panel of Figure~\ref{fig:dists} (solid line).

For the color distribution, in addition to a simple shift, we also
quantify the effect of including a smaller or larger fraction of SNe
significantly reddened by dust. In fact, we have good reasons to
believe that most cluster SNe~Ia will be in dust-free environments. A
large fraction of the stellar mass in the clusters ($\sim 80\%$) is
contained in red-sequence galaxies expected to have little or no
dust. Our spectroscopic and photometric analysis \citep{meyers11a} of
the red-sequence galaxies confirms this expectation. Therefore, for
our default $c$ distribution (Fig.~\ref{fig:dists}, right panel, solid
line), we assumed that $20\%$ of SNe (those occurring in galaxies not
on the red sequence) could be affected by dust, and that the
extinction of these SNe would be distributed according to
$P(A_V) \propto \exp(-A_V/0.33)$. This distribution is based on the
inferred underlying $A_V$ distribution of the SDSS-II
sample \citep[][hereafter K09]{kessler09a}. All SNe are assumed to
have an intrinsic dispersion in color to match the observed SNLS
distribution at $c<0.3$. It might be the case that even fewer SNe are
affected by dust, or (unlikely) more SNe are affected by dust. As
extreme examples, we tested two alternative distributions (dotted
lines in Fig.~\ref{fig:dists}). In the first, we assumed that the SNLS
sample was complete and characterized the full $c$ distribution, with
a negligible number of $c>0.4$ SNe. This increases the control time by
only $2\%$. In the second, we increase the fraction of dust-affected
SNe from $20\%$ to $50\%$.  Even though this alternative distribution
includes an additional $\sim$$30\%$ more reddened SNe (unlikely to be
true in reality), the average control time is only lower by $9\%$
(increasing the rate by $10\%$). We use these values as the systematic
error in the assumed dust distribution.


%The assumed distribution of SN color used in the Monte Carlo
%simulation was based on the SNLS $z<0.6$ sample, which contained no
%SNe redder than $c=0.3$. As a result, our assumed $c$ distribution has
%only a very small number of SNe with $c>0.4$. There are two reasons
%that we expect this to be a good representation of reality. First, the
%SNLS sample is reasonably complete at $z<0.6$; if there were many
%$c>0.4$ SNe in reality, some would be found at lower
%redshifts. Note that the trend of no detected $c>0.4$ SNe continues
%in the third-year SNLS sample \citep{sullivan10a}. Second, the stellar
%mass in clusters is predominantly found in early-type galaxies which
%are expected to have less dust than field galaxies. Of course, this
%doesn't mean there are {\it no} high-extinction SNe~Ia in the clusters;
%high-extinction SNe~Ia have been detected in other surveys, and clusters
%do contain some dusty galaxies. However it does mean that there are
%likely a negligible number of $c>0.4$ SNe~Ia occurring in the clusters.

%Still, to test the effect of the existence of high-extinction SNe on
%our control time, we replace our assumed $c$ distribution with one
%including a number of redder SNe. We conservatively assumed that
%$50\%$ of cluster SNe could be affected by dust, and that the
%extinction of these SNe would be distributed according to
%$P(A_V) \propto \exp(-A_V/0.33)$. This is the inferred underlying
%$A_V$ distribution of the SDSS-II sample \citep{kessler09a}. We
%convert this to a distribution in $c$ ($c = A_V/(\beta-1)$) and add
%intrinsic SN color scatter to reach the distribution shown in
%Fig.~\ref{fig:dists} (dotted line). This distribution roughly matches
%our nominal one in the range $-0.2 < c < 0.3$, but includes an
%additional $\sim$$40\%$ more reddened SNe. Using this alternate
%distribution in our simulation, the average control time is lower by
%$10.5\%$. (The effect is greater for higher-redshift clusters and
%smaller for lower-redshift clusters, where SNe are more easily
%detected.) Across all clusters, we would detect the majority of the
%additional $40\%$ more-reddened SNe. Conservatively, we include a
%one-sided control time systematic error of $10.5\%$ in our result.
