The 86 remaining candidates still include a considerable number of
non-SNe. We wish to trim the sample down as much as possible in an
automated way, so that we can easily calculate the efficiency of our
selection.  For each candidate, we now make three further automated
requirements based on $i_{775}$ data (if available) and the shape of
the $z_{850}$ light curve.  The requirements and number of candidates
remaining after each requirement are summarized in
Table~\ref{tab:lccuts}.

%%%%%%%%%%%%%%%%%%%%
% Light curve Cuts %
%%%%%%%%%%%%%%%%%%%%
\begin{table}[tbh]
\caption{\label{tab:lccuts} Light curve requirements for candidates}
\begin{center}
\begin{footnotesizetabular}{lc}

\hline
\hline
Requirement & Candidates Remaining \\
\hline
Before light curve requirements                   & 86\\
Positive $i_{775}$ flux (if observed in $i_{775}$) & 81\\
$2\sigma$ Detection in surrounding epochs         & 73\\
If declining, Require two $5\sigma$ detections    & 60\\
\hline

\end{footnotesizetabular}
\end{center}
\end{table}

First, we require that if $i_{775}$ data exists for the epoch in which
the candidate was detected, there be positive flux in a 2~pixel radius
aperture at the candidate location in the $i_{775}$ image. From our SN
light curve simulations, we find that virtually all SNe should pass
(near maximum light there is typically enough SN flux in the $i_{775}$
filter to result in a positive total flux, even with large negative
sky fluctuations).  Meanwhile, about half of the cosmic rays located
far from galaxies will fail this test (due to negative sky
fluctuations). If there is no $i_{775}$ data for the detection epoch,
this requirement is not applied. Even though nearly all SNe are
expected to pass, we account for any real SNe that would be removed in
our Monte Carlo simulation.

Second, we require that the light curve does not rise and fall too
quickly: if there is a ``search'' visit less than 60~days before the
detection visit and also one less than 60~days after the detection visit,
the candidate must be detected at a $2\sigma$ level in at least one of
these two visits. SNe~Ia have light curves wide enough to be detected
at this level in two epochs spaced apart by 60~days. However, cosmic
rays in one $z_{850}$ image are unlikely to be repeated in the same
spot in two epochs and thus will be removed. This requirement is also
included in our Monte Carlo simulation.

%The third and final requirement eliminates SNe lacking
%sufficient light curve information for typing. 
%Because we cannot
%obtain spectroscopic confirmation for all candidates, it is important
%to have enough light curve information to be able the estimate a SN
%type photometrically. 
The third and final requirement aims to eliminate candidates that were
significantly detected in only the first epoch and that then faded
from view. Such candidates would not have been followed up
spectroscopically and it would typically be impossible to tell if such
candidates were SNe (and if so, Type~Ia or core collapse) on the basis
of a single detection. We chose to eliminate any such candidates and
account for this elimination in our Monte Carlo simulation, rather
than dealing with an ``untypeable'' candidate.  Specifically, if a
candidate is found on the decline (in the first search epoch), we
require two epochs with $5\sigma$ detections.  For high-redshift
($z \sim 1$) SNe~Ia, this requirement means that the first epoch will
be at approximately maximum light, and most of the SN decline is
captured, making it possible to confirm a SN and estimate a type.
For candidates that are only significantly detected in the last search
epoch, typing is not a problem because additional ACS orbits were
typically scheduled in order to follow such candidates.

After these requirements 60 candidates remain. The automatic selection
means that we can easily calculate the completeness of the selection
so far; any real SNe~Ia removed will be accounted for in the
``effective visibility time'' which is calculated using a
Monte Carlo simulation.
