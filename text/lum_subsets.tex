
In addition to measuring the total luminosity of all galaxies in the
clusters, we also measure the total luminosity of only red-sequence
galaxies and the total luminosity of only red-sequence, morphologically
early-type galaxies. These measurements enable us to compute the
cluster SN~Ia rate specifically in these galaxy subsets. For the
red-sequence-only measurement we follow the same procedure as above,
but eliminate from the analysis all galaxies with $i_{775} - z_{850}$
colors more than 0.2~mag from their respective cluster red sequences
\citep[galaxy colors and cluster red sequences are determined as in][]
{meyers11a}. For the red-sequence early-type measurement, we make the
same requirement in color, and additionally use the quantitative
morphology requirements of \citet{meyers11a}. That analysis uses two
parameters, asymmetry and Gini coefficient, to automatically divide
galaxies into early- and late-type subsets. Here we require the
asymmetry to be $<0.10$ and the Gini coefficient to be $>0.40$. We
also require the galaxies to be $z_{850} < 24$ as the asymmetry and
Gini coefficient are somewhat less reliable at fainter magnitudes.

The luminosity profiles for these two subsets are shown in the center
and right columns of Figure~\ref{fig:avgprofile_mult}. The profiles
are broadly consistent with the profile of the full cluster luminosity
(left column), but the ``subset'' profiles are much better
measured. This is because by excluding bluer galaxies, we have
eliminated much of the background while still retaining the majority
of cluster galaxies. The red-sequence subset contains $77\%$ of the
luminosity of the full cluster within $0.6$~Mpc
(Table~\ref{tab:lum_avg}).  The red-sequence early-type subset has
$62\%$ of the light contained in the red-sequence subset. However,
keep in mind that in the early-type subset we have excluded
$z_{850}>24$ galaxies, whereas they are included in the red-sequence
subset: In fact $68\%$ of $z_{850}>24$ red-sequence galaxies pass the
``early-type'' morphology requirements.

Note that our definition of ``red-sequence'' here is a relatively
simple one. It is sufficient to select a subsample of ``more red''
galaxies for the purpose of looking for a dependence of the SN rate
with galaxy color within the cluster. However, for measuring the red
fraction in clusters
\citep[e.g., the Butcher-Oemler effect][]{butcher78a,butcher84a}, 
defining red-sequences with a constant color width for all redshifts
is not ideal \citep{andreon06d}. The luminosity content of the subsets
are reported above only to give the relative size of each sample.

%The cluster sample is also divided by discovery subsets and redshift
%subsets, although differences between the redshift subsets are
%strongly correlated with differences in discovery subsets, as the
%optical-discovered clusters dominate the lower-redshift sample.
