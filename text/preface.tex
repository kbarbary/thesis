
Cosmology has come a long way in the last two decades. At the
beginning of the 1990s there were large uncertainties regarding the
age of the universe, whether it is flat or curved, and even the nature
of its major components. Today, we have overwhelming confidence that
we live in a flat, accelerating universe dominated by dark energy and
we have moved on to measuring its parameters with percent-level
accuracy.

Much of this advance has been thanks to Type Ia supernovae (SNe~Ia), a
type of stellar explosion that always has (more or less) the same
intrinsic brightness. A painstakingly acquired handful of these
supernovae was used to determine that the universe is
accelerating. Cosmologists have now become experts at finding them and
using them as distance indicators at the largest scales. The current
world sample of well observed supernovae has surpassed one thousand,
spanning from those in the local universe out to supernovae that
exploded over 9 billion years ago.

In spite of these advances, there is still much we don't know about
how these explosions occur. As the field pushes forward, a more
complete understanding of supernovae is becoming more and more
important for measuring cosmological parameters with the accuracy
needed to distinguish between models for dark energy. The work in this
thesis is a small step towards a better understanding of Type Ia
supernovae, via a measurement of the rate at which they occur.

This work is based on a survey carried out by the Supernova Cosmology
Project (SCP) using the \emph{Hubble Space Telescope (HST)} during
2005 and 2006, called the \emph{HST} Cluster Supernova Survey. The
main aim of the survey was to improve both the efficiency and
usefulness of high-redshift SN observations with \emph{HST} by
specifically targeting high-redshift galaxy clusters. Final results
from the survey are now coming to fruition, with a total of ten
publications related to the supernova work and ten more (as of this
writing) related to the cluster studies. This thesis represents my
analyses of the data from the survey. These analyses have also been
presented in \citet{barbary09a,barbary11a}, and will also be the
subject of a third article, in preparation.

This thesis begins in Chapter~\ref{sec:intro} with a review of SN~Ia
progenitor models and work that has been done to differentiate between
them using SN~Ia rates. Chapter~\ref{sec:survey} describes the
\emph{HST} Cluster Supernova Survey, placing particular emphasis 
on the aspects relevant to the rate calculation. During the survey we
discovered a very unusual transient. This was the subject of
{\bf \citet{barbary09a}} and is discussed here in
Chapter~\ref{sec:scp06f6}. Chapter~\ref{sec:cands} lays out the
systematic selection of supernova candidates used in the rate
calculations and the determination of supernova type for these
candidates {\bf \citep[first part of][]{barbary11a}}.  In
Chapter~\ref{sec:clrate} the cluster SN~Ia rate is calculated based on
the candidates in the clusters and, using this rate, the SN~Ia delay
time distribution is calculated {\bf \citep[second part
of][]{barbary11a}}. This is followed by a calculation of the
volumetric field rate based on the non-cluster-member candidates in
Chapter~\ref{sec:fieldrate} {\bf (Barbary et al., in preparation)}.
Finally, the thesis work is summarized in Chapter~\ref{sec:conclusions}.
