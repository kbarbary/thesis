
With the systematically selected SN sample from the previous chapter,
we are now in a position to calculate SN rates. The cluster SN~Ia rate
is given by
\begin{equation}
\label{eq:rate}
\mathcal{R} = \frac{N_{\rm SN~Ia}}{\sum_j T_j L_j} ,
\end{equation}
where $N_{\rm SN~Ia}$ is the total number of SNe~Ia observed in
clusters in the survey, and the denominator is the total effective
time-luminosity for which the survey is sensitive to SNe~Ia in
clusters. $L_j$ is the luminosity of cluster $j$ visible to the survey
in a given band. $T_j$ is the ``effective visibility time'' (also
known as the ``control time'') for cluster $j$. This is the effective
time for which the survey is sensitive to detecting a SN~Ia,
calculated by integrating the probability of detecting a SN~Ia as a
function of time over the span of the survey. It depends on the
redshift of the SN~Ia to be detected and the dates and depths of the
survey observations. As each cluster has a different redshift and
different observations, the control time is determined separately for
each cluster.  To calculate a rate per stellar mass, $L_j$ is replaced
by $M_j$.

Equation~(\ref{eq:rate}) is for the case where the entire observed
area for each cluster is observed uniformly, yielding a control time
$T$ that applies to the entire area.  In practice, different areas of
each cluster may have different observation dates and/or depths,
resulting in a control time that varies with position. This is
particularly true for this survey, due to the rotation of the observed
field between visits and the gap between ACS chips. Therefore, we
calculate the control time as a function of position in each observed
field, $T_j(x,y)$. As the cluster luminosity is also a function of
position, we weight the control time at each position by the
luminosity at that position. In other words, we make the substitution
\begin{equation} 
\label{eq:ratedenom}
T_j L_j \Rightarrow \int_{x,y} T_j (x,y) L_j (x,y). 
\end{equation}

In \S\ref{sec:clrate_candsummary} we summarize the findings of the
previous chapter for $N_{\rm SN~Ia}$.  In \S\ref{sec:ct} we calculate
$T_j (x,y)$, and in \S\ref{sec:lum} we calculate $L_j(x,y)$.

