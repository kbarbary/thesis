
The effective visibility time $T$ at a position $(x,y)$ on the 
sky is given by
\begin{equation}
T(x,y) = \int_{t=-\infty}^{t=\infty} \eta^\ast (x,y,t) \epsilon (x,y,t) dt.
\end{equation}
The integrand here is simply the probability for the survey and our
selection method to detect (and keep) a SN~Ia at the cluster redshift
that explodes at time $t$, and position $(x,y)$. This probability is
split into the probability $\eta^\ast$ of detecting the supernova and
the probability $\epsilon$ that the supernova passes all ``light
curve'' cuts. As each SN has multiple chances for detection, the total
probability of detection $\eta^\ast$ is a combination of the
probabilities of detection in each observation. For example, if we
have two search visits at position $(x,y)$, $\eta^\ast(t)$ is given by
\begin{equation}
\eta^\ast (t) = \eta_1 (t)+ ( 1-\eta_1(t) ) \eta_2 (t),
\end{equation}
where $\eta_i (t)$ is the probability of detecting a SN~Ia 
exploding at time $t$ in visit $i$. In other words, the total 
probability of finding the SN~Ia exploding at time $t$ is the probability 
of finding it in visit 1 plus the probability that it was \emph{not} found in
visit 1 times the probability of finding it in visit 2. This can be 
generalized to many search visits: The contribution of each additional visit 
to the total probability is the probability of not finding the SN in any 
previous visit times the probability of finding the SN in that visit.

In practice, we calculate $T(x,y)$ in two steps: First, we determine
the probability $\eta$ of detecting a new point source in a single
image as a function of the point source magnitude. This is discussed
in \S\ref{sec:ct_eff}. Second, for each $(x,y)$ position in the observed area
we simulate a variety of SN~Ia light curves at the cluster redshift
occurring at various times during the survey. By considering the dates
of the observations made during the survey at that specific position,
we calculate the brightness and significance each simulated SN~Ia would
have in each $z_{850}$ and $i_{775}$ image. We then use our
calculation of $\eta$ as a function of magnitude to convert the
observed brightness into a probability of detecting the simulated SN
in each observation. The light curve simulation is discussed in
\S\ref{sec:ct_mc}. 
