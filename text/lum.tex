
In this section, we calculate the total luminosity of each cluster and
use the luminosity to infer a stellar mass. Only a small subset of
galaxies in each field have known redshifts, making it impossible to
cleanly separate cluster galaxies from field galaxies.  Therefore, we
use a ``background subtraction'' method to estimate cluster
luminosities statistically: we sum the luminosity of all detected
galaxies in the field and subtract the average ``background
luminosity'' in a non-cluster field.  This approach follows that
of \citet{sharon07a}.  For the blank field, we use the
GOODS\footnote{Based on observations made with the NASA/ESA 
\emph{Hubble Space Telescope}. The observations are associated with programs
GO-9425, GO-9583 and GO-10189} \citep{giavalisco04a} fields as they
have similarly deep or deeper observations in both ACS $i_{775}$ and
$z_{850}$.

In \S\ref{sec:lum_background} we describe the estimation of image
backgrounds, which must be subtracted to avoid biasing photometry
measurements. \S\ref{sec:lum_photometry} describes the galaxy
detection and photometry method.  Simply summing the photometry from
the detected galaxies would include most of the total cluster
light. However, for an unbiased estimate of the total light, several
small corrections are necessary: We account for light in the outskirts
of each galaxy (\S\ref{sec:lum_correction}), and light from faint
galaxies below the detection threshold
(\S\ref{sec:lum_faintgals}). These corrections are on the order of
20\% and 5\% respectively. In \S\ref{sec:lum_kcorr} we convert the
observed $z_{850}$ flux to a rest-frame $B$-band
flux. \S\ref{sec:lum_centers} describes the method for determining the
center of a cluster. In \S\ref{sec:lum_profiles} we sum the light and
subtract light from non-cluster galaxies, creating a profile of
cluster light as a function of radius. In \S\ref{sec:lum_subsets} we
repeat this calculation limiting ourselves to red-sequence and
red-sequence early-type subsets of galaxies. Finally,
in \S\ref{sec:lum_mass} we estimate cluster stellar masses based on
the cluster luminosities and stellar mass-to-light ratios.
