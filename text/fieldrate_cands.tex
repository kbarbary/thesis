

For $N_{\rm SN~Ia}$ we use the SN selection from
Chapter~\ref{sec:cands} (the non-cluster-members in
Table~\ref{tab:sn}) with two additional selections not used in the
cluster rate analysis:

(1) First, we eliminate candidates that could only be consistent with
a SN~Ia if it peaked prior to 10 rest-frame days before the first
observation. We found that lower-redshift ($z \lesssim 0.9$) SNe were
detectable even when peaking well before the first observation, but
that such SNe were extremely difficult to type as they were observed
only far into the light curve decline. We found it most ``fair'' to
eliminate such candidates entirely. We include the same selection in
our efficiency simulations below. This selection affects candidates
SCP06L21 and SCP05N10. This was not an issue for the cluster rate
analysis because SNe of interest (at $z \ge 0.9$) are not detectable
very far after peak.

(2) Second, we exclude regions within $20''$ of cluster centers, in
order to avoid the most strongly lensed areas in the volume behind the
clusters. This region is only $\sim$$3\%$ of the observed field of
each cluster. Note that we were careful to choose this radius before
looking at the radii of any of the candidates, in order to avoid
biasing ourselves by adjusting the radius to conveniently exclude or
include candidates. Two candidates were excluded as a result: SNe
SCP06B3 ($16.8''$ from the cluster center) and SCP06M50 ($19.4''$ from
the cluster center). As it happens, these candidates are unlikely to
be SNe~Ia. SN SCP06B3 is a ``probable'' SN~CC, while SN SCP06M50 is
possibly not a SN at all and may be hosted by a cluster member galaxy,
making its position near the cluster center unsurprising.  The
exclusion of this region is taken into account in our simulations
(\S\ref{sec:fieldrate_ct}). The effect of lensing on the remaining
portions of the fields are discussed in \S\ref{sec:lensing}.

The systematic uncertainties associated with the determination of SN
type and redshift for the remaining candidates are addressed
in \S\ref{sec:systyping}.

