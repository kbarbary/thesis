The central work in this thesis has been a calculation of SN~Ia rates
from the \emph{HST} Cluster Supernova Survey. This required the
systematic selection of SN candidates and determination of their
types. As part of this selection and typing we have presented data on
the unusual transient SCP06F6, most likely to be a new rare type of
SN.  Here, we highlight the unique aspects of the measurements
themselves and draw conclusions about the cluster and field rate
results.

\section{Measurements}

We have benefited from an unusually complete dataset (particularly for
a cluster rate study). As a result, the measurements are quite
robust. For the cluster rate measurement in particular, statistical
and systematic uncertainties are on par with or better than
measurement uncertainties at low redshift. We highlight several
important and/or unique aspects of the measurements:
\begin{itemize}
\item The SN classification approach takes advantage
  of all relevant information. Thanks to the ``rolling search''
  strategy of the survey and the nearly complete spectroscopic
  follow-up, most candidates have a full light curve and a host galaxy
  redshift, greatly reducing classification uncertainty.
\item The position-dependent control time allows one to calculate a
  supernova rate given an arbitrary observing pattern and luminosity
  distribution.
\item The control time calculation includes a full distribution of SN
  properties and the systematic uncertainty associated with the assumed 
  distribution is carefully quantified. For the cluster rate, thanks to the
  depth of the observations, the detection efficiency approaches
  100\% during the period of the survey for most of the clusters,
  meaning that the systematic uncertainty is low.
\item For the cluster rate, statistical uncertainties associated with the cluster
  luminosities, including both statistical variations and cosmic
  variance, are included in the total uncertainty.  Also, light in the
  outskirts of each galaxy (outside the {\sc SExtractor} MAG\_AUTO
  aperture) is accounted for.  This is a significant component of the
  total cluster luminosity.
\item Cluster SN~Ia rate measurements are normalized consistently across
  redshifts using a redshift-dependent mass-to-light versus color
  relation.
\end{itemize}

\section{Cluster Rate}

For the first time, our result shows (at the $>2\sigma$ level) that
the cluster SN~Ia rate is increasing with redshift.  Simply by
comparing the low- and high-redshift cluster rate measurements, the
shape of the late-time SN~Ia delay time distribution can be
constrained. The power of the measurement for this purpose comes both
from the high redshift and relatively low statistical and systematic
uncertainties in the measurement. While we cannot conclusively rule
out either the single degenerate or double degenerate class of
progenitors via the delay time distribution, the binary evolution that
could lead to each model are constrained. The DD scenario is
consistent with the measurement under a wide range of plausible binary
evolution parameters, while there is a stronger constraint on binary
scenarios that could lead to an SD scenario. Finally, this measurement
is unique in constraining the delay time distribution at delay times
of a few Gyr. In future studies, it can be used in combination with
other cluster rates and other delay time distribution
measurements \citep[e.g.,][]{maoz10c} to place even tighter
constraints on models for binary evolution and SN~Ia progenitor
scenarios.

\section{Field Rate}

We computed volumetric SN~Ia rates based on $\sim$12 SNe~Ia discovered
in 189 \emph{HST} orbits. This large \emph{HST} dataset adds
significant statistics to the existing \emph{HST} rate measurements,
previously based only on the GOODS fields. The availability of raw
data from our efficiency simulations makes it simple to combine this
dataset with current and future \emph{HST} datasets, such as the
in-progress CANDELS survey.

We find that the dominant systematic uncertainty in our result is the
amount of host-galaxy dust assumed in our simulations. In fact,
differences in these assumptions can explain a large amount of the
discrepancy between different \emph{HST} GOODS rate measurements and
differences between \emph{HST} and ground-based measurements. This
illustrates the need to use caution in calculating and interpreting SN
rate results, especially as statistical error decreases and systematic
uncertainties become dominant.

\section{Status and Future Work}

The measurements presented in this thesis are just part of a greater
transition toward an accurate measurement of the SN~Ia delay time
distribution. When the \emph{HST} Cluster Supernova Survey was
beginning five years ago, little was known about the shape of the DTD
from observations. Since then, the situation has changed
dramatically. Results using a wide variety of methods, including
volumetric rates and cluster rates, have coalesced into a consistent
picture of a DTD declining from a few hundred Myr to $\sim$10~Gyr,
with a slope consistent with $\sim$$t^{-1}$. The cluster rate
measurements presented here have been instrumental in mapping out the
DTD in the longer delay time regions.

Still, we have only yet made the coarsest measurement of the SN~Ia
DTD: The data are good enough to fit a power-law slope and an
amplitude, but not much more. There is much we can still learn from a
more accurate measurement. Does the DTD truly follow a single power
law over the full range of delay times? If so, is the slope really
exactly $t^{-1}$? There is no reason to think the answer to either
question is yes. By finding a more detailed behavior and/or a more
precise power law slope, we can learn much more about not only the
SN~Ia progenitor scenario, but also details of binary evolution.

Going forward, there are still many gains to be made simply from
increased statistics. Specifically for cluster rate measurements,
Poisson statistics dominate the uncertainty in the measurement
presented in this thesis (and as a result, the DTD constraints as
well). Forthcoming results from the Multi-Epoch Nearby Cluster
Survey \citep{sand08a,sand10a} will increase statistics at $z \lesssim
0.15$ with 22 new cluster SN~Ia discoveries. At $z \sim 0.5$, the
\emph{HST} CLASH survey\footnote{\url{http://www.stsci.edu/~postman/CLASH}} will
provide on the order of 10 -- 15 new cluster SNe~Ia. At the highest
redshifts, another program similar to the \emph{HST} Cluster Supernova
Survey, or perhaps a ground-based dedicated cluster rate survey on a
10m class telescope, could enhance statistics in this crucial redshift
regime.

In other DTD measurements, systematics are already playing a dominant
role. For field rate measurements, \citet{graur11a} notes that
uncertainties in the cosmic SFH are already the dominant uncertainty
in determining the DTD. In this thesis we have seen that uncertainty
in the host galaxy dust distribution is also significant compared to
the statistical uncertainty in rate measurements. In the future,
deeper surveys can help limit this uncertainty by directly
constraining the numbers of extincted SNe~Ia. The
in-progress \emph{HST} CANDELS
survey\footnote{\url{http://candels.ucolick.org}} will do this to some
extent, discovering SNe~Ia well past $z=2$, but much more significant
gains will be possible with a dedicated wide-field infrared survey
telescope.


