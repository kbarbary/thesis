
This thesis presents Type~Ia supernova (SN~Ia) rates from
the \emph{Hubble Space Telescope (HST)} Cluster Supernova Survey, a program
designed to efficiently detect and observe high-redshift supernovae by
targeting massive galaxy clusters at redshifts $0.9 < z <1.46$. Among
other uses, measurements of the rate at which SNe~Ia occur can be used
to help constrain the SN~Ia ``progenitor scenario.''  The progenitor
scenario, the process that leads to a SN~Ia, is a particularly poorly
understood aspect of these events. Fortunately, the progenitor is
directly linked to the delay time between star formation and supernova
explosion. Supernova rates can be used to measure the distribution of
these delay times and thus yield information about the elusive
progenitors.

Galaxy clusters, with their simpler star formation histories, offer an
ideal environment for measuring the delay time distribution. In this
thesis the SN~Ia rate in clusters is calculated based on $8 \pm 1$
cluster SNe~Ia discovered in the \emph{HST} Cluster Supernova
Survey. This is the first cluster SN~Ia rate measurement with detected
$z>0.9$ SNe. The SN~Ia rate is found to be $0.50^{+0.23}_{-0.19}$
(stat) $^{+0.10}_{-0.09}$ (sys) $h_{70}^2$ SNuB (SNuB $\equiv
10^{-12}$ SNe~$L_{\odot,B}^{-1}$~yr$^{-1}$), or in units of stellar
mass, $0.36^{+0.16}_{-0.13}$ (stat) $^{+0.07}_{-0.06}$ (sys)
$h_{70}^2$ SNuM (SNuM $\equiv 10^{-12}$
SNe~$M_\odot^{-1}$~yr$^{-1}$). This represents a factor of $\approx
5 \pm 2$ increase over measurements of the cluster rate at $z<0.2$ and
is the first significant detection of a changing cluster SN~Ia rate
with redshift. Parameterizing the late-time SN~Ia delay time
distribution with a power law ($\Psi(t) \propto t^s$), this
measurement in combination with lower-redshift cluster SN~Ia rates
constrains $s = -1.41^{+0.47}_{-0.40}$, under the approximation of a
single-burst cluster formation redshift of $z_f = 3$.  This is
generally consistent with expectations for the ``double degenerate''
progenitor scenario and inconsistent with some models for the ``single
degenerate'' progenitor scenario predicting a steeper delay time distribution at
large delay times.  To check for environmental dependence and the
influence of younger stellar populations the rate is also calculated
specifically in cluster red-sequence galaxies and in morphologically
early-type galaxies, with results similar to the full cluster
rate. Finally, the upper limit of one host-less cluster SN~Ia detected
in the survey implies that the fraction of stars in the intra-cluster
medium is less than 0.47 ($95\%$ confidence), consistent with
measurements at lower redshifts.

The volumetric SN~Ia rate can also be used to constrain the SN~Ia
delay time distribution. However, there have been discrepancies in
recent analyses of both the high-redshift rate and its implications
for the delay time distribution.  Here, the volumetric SN~Ia rate out
to $z \sim 1.6$ is measured, based on $\sim$12 SNe~Ia in the
foregrounds and backgrounds of the clusters targeted in the survey.
The rate is measured in four broad redshift bins. The results are
consistent with previous measurements at $z \gtrsim 1$ and strengthen
the case for a SN~Ia rate that is $\gtrsim$$0.6 \times 10^{-4}
h_{70}^{3}$~yr$^{-1}$~Mpc$^{-3}$ at $z \sim 1$ and flattening out at
higher redshift. Assumptions about host-galaxy dust extinction used in
different high-redshift rate measurements are examined. Different
assumptions may account for some of the difference in published
results for the $z \sim 1$ rate.
