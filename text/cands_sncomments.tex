
Here we comment in greater detail on a selection of individual
candidates, particularly those with the greatest uncertainty in
typing. For each candidate, see the corresponding panel of
Figure~\ref{fig:sn} for an illustration of the candidate host galaxy and
light curve.

%{\it SN SCP06F12} is the only cluster-member SN lacking both
%spectroscopic confirmation and an early-type host galaxy, and is thus
%the only cluster-member SN with significant typing uncertainty. The
%light curve is consistent with both SN~Ia and SN Ibc templates (see
%Fig.~\ref{fig:sn}) and to a lesser extent SN II-L. However, the
%best-fit absolute magnitude ($M_B=-19.2$) and color ($c=-0.1$) for the
%SN~Ia template are well within the normal ranges for SNe~Ia, while
%they are fairly extreme for SNe Ibc and II-L. Thus, we classify
%SCP06F12 as a ``plausible'' SN~Ia.

{\it SN SCP06E12}. We were unable to obtain a host galaxy redshift due
to the faintness of the host. The color of the host galaxy is
consistent with the cluster red sequence. The candidate light curve is
consistent with a SN~Ia at the cluster redshift of $z=1.03$, but is
also consistent with SN II-L at $z=1.03$. Different SN types provide
an acceptable fit over a fairly wide range of redshifts. As the SN~Ia
template provides a good fit with typical parameters, we classify the
candidate as a ``plausible'' SN~Ia.  However, there is considerable
uncertainty due to the uncertain redshift.


{\it SN SCP06N32} also lacks a host galaxy redshift. If the cluster
redshift of $z=1.03$ is assumed, the candidate light curve is best fit
by a SN~Ibc template. A SN~Ia template also yields an acceptable
fit, but requires an unusually red color of $E(B-V) \sim 0.6$. Given
the best-fit $s$ and $M_B$ values, the candidate would have an
unusually large Hubble diagram residual of approximately
$-0.8$~magnitudes. If the redshift is allowed to float, a SN~Ia
template with more typical parameters provides an acceptable fit at $z
= 1.3$. A SN~Ibc template still provides a better fit, with the best
fit redshift being $z \sim 0.9$. As SN~Ibc
provides a better fit in both cases, we classify this as a ``plausible''
SN~CC. However, there is considerable uncertainty in
both the type and cluster membership of this candidate. 

{\it SN SCP06A4}. We note that this candidate was observed
spectroscopically, as reported in Dawson09. While the spectrum was
consistent with a SN~Ia, there was not enough evidence to
conclusively assign a type. The host galaxy is morphologically and
photometrically consistent with an early-type galaxy, but there is
detected [O{\sc ii}], a possible indication of star formation. We therefore
rely on light curve typing for this candidate, assigning a confidence
of ``probable'' rather than ``secure.''
 
{\it SN SCP06G3} has only sparse light curve coverage. The best fit
template is a SN~Ia with $s=1.3$, $E(B-V)=0.3$ and $M_B=-18.5$, although
these parameters are poorly constrained. A large stretch and red
color would not be surprising given the spiral nature of the host
galaxy. It is also consistent with a II-L template, although the best
fit color is unusually blue: $E(B-V)=-0.1$. Given that SN~Ia yields
more ``typical'' fit parameters and that, at $z \sim 1$ a detected SN
is more likely to be Type~Ia than II, we classify this as a
``plausible'' Type Ia, with considerable uncertainty in the type.

%{\it SN SCP06F12} is the only cluster-member SN lacking both
%spectroscopic confirmation and an early-type host galaxy, and is thus
%the only cluster-member SN with significant typing uncertainty. The
%light curve is consistent with both SN~Ia and SN Ibc templates (see
%Fig.~\ref{fig:sn}) and to a lesser extent SN II-L. However, the
%best-fit absolute magnitude ($M_B=-19.2$) and color ($c=-0.1$) for the
%SN~Ia template are well within the normal ranges for SNe~Ia, while
%they are fairly extreme for SNe Ibc and II-L. Thus, we classify
%SCP06F12 as a ``plausible'' SN~Ia.

%{\it SN SCP06B3} is too dim ($M_B>-18$) to fit a SN~Ia. Furthermore,
%the $i_{775}$ light curve is much broader than the best-fit SN~Ia. Both
%SN II-P and II-L templates provide a better fit, with regard to shape,
%absolute magnitude and color. \NOTE{GA: Host type?}

%{\it SN SCP06C7} has a light curve too dim, too blue, and too narrow to
%fit a SN~Ia template, even with $s=0.6$, $c=-0.2$ and $M_B=-18.0$
%(shown in Fig.~\ref{fig:sn}). An SN II-L light curve is a far better fit
%with $M_B=-17.2$, $E(B-V)=0.0$. \NOTE{GA: Host type?}

%{\it SN SCP06F8} also has a light curve that clearly favors a SN CC. A
%dim SN~Ia with $M_B=-18.0$ and $s=0.7$ can fit the peak of the
%light curve, but fades too fast in $z_{850}$ to fit the late-time
%light curve. An SN II-P with $M_B=-17.3$, $E(B-V)=0.4$ provides a much
%better fit. For all three of the above SNe, SN SCP06B3, SN SCP06C7 and
%SN SCP06F8, because the light curve clearly favors CC SNe and disfavors
%SNe~Ia, we regard these as ``probable'' SNe CC.

{\it SN SCP06L21} lacks a spectroscopic redshift, but has a distinct
slowly-declining light curve that rules out a $z>0.6$ SN~Ia light
curve. Even the best-fit Ia template at $z=0.55$, shown
in Fig.~\ref{fig:sn}), is unusually dim ($M_B \approx -17.5$), making it
unlikely that the candidate is a lower-redshift SN~Ia. The light curve
is better fit by a SN~II-P template (with the best-fit redshift being
$z=0.65$). We therefore classify the candidate as a ``probable''
SN~CC.

{\it SN SCP06M50} is the most questionable ``SN'' candidate, having no
obvious $i_{775}$ counterpart to the increase seen in $z_{850}$. It
may in fact be an image artifact or AGN. However, it appears to be off
the core of the galaxy by $\sim$2~pixels (making AGN a less likely
explanation), and shows an increase in $z_{850}$ flux in two
consecutive visits, with no obvious cosmic rays or hot pixels (making
an image artifact less likely as well). The galaxy is likely to be a
cluster member: its color and magnitude put it on the cluster red
sequence, it is morphologically early-type, and it is only $19''$ from
the cluster center. Under the assumption that the candidate is a
supernova and at the cluster redshift of $z=0.92$, no template
provides a good fit due to the lack of an $i_{775}$ detection and the
constraints on $E(B-V)$. In particular, a SN~Ia template would
require $E(B-V)>0.6$. (The best-fit template shown in
Fig.~\ref{fig:sn} is with $E(B-V) = 0.6$.) If the redshift is allowed
to float, it is possible to obtain a good fit at higher redshift
($z \sim 1.3$), but still with $E(B-V) \gtrsim 0.4$, regardless of the
template type. Given the color and early-type morphology of the host
galaxy, it is unlikely to contain much dust. There is thus no
consistent picture of this candidate as a SN, and we do not assign a
type. However, note that the candidate is unlikely to be a cluster
SN~Ia.

{\it SN SCP05N10} is the lowest-redshift SN candidate in our sample at
$z=0.203$. Its light curve shape is inconsistent with a SN~Ia occurring
well before the first observation, and its luminosity is too low for
a SN~Ia with maximum only slightly before the first
observation. Therefore, we call this a ``probable'' SN CC. For all SN
types, the best fit requires maximum light to occur well before the
first observation, making all fits poorly constrained.

%{\it SN SCP06N33} is fit quite well by a SN~Ia template with typical
%parameters $c=-0.1$, $s=0.8$, and $M_B=-19.1$. The light curve is too
%blue to be fit by a SN~Ibc template, even with $c=-0.1$. An SN~II-P
%fits near maximum light with $c=-0.1$, but does not decline fast
%enough to fit the last data point. Because a SN~Ia template provides
%a much better fit than other templates, we call this a ``probable''
%SN~Ia.

%{\it SN SCP05P1} has a well-sampled light curve in $z_{850}$, but is
%lacking $i_{775}$ data during the peak of the light curve. An SN~Ia is
%an excellent fit to the data, with $s=1.1$, and $M_B=-19.1$. The only
%SN CC template that also fits the data is a SN Ibc. As the
%parameters are more typical of a SN~Ia (and SNe~Ia are much more
%likely to be discovered at this redshift), we classify this a
%``plausible'' SN~Ia.

%{\it SN SCP06U7} has a slowly-declining light curve, poorly fit by an
%SN~Ia template (see Fig.~\ref{fig:sn}). An SN II-P is the best fit
%template, but requires a color bluer than the template
%($E(B-V)=-0.1$). Because the light curve shape is clearly inconsistent
%with a SN~Ia, we classify this as a ``probable'' SN CC.

{\it SN SCP06X26} has a tentative redshift of $z=1.44$, derived from a
possible [O{\sc ii}] emission line in its host galaxy. Given this redshift,
a Ia template provides an acceptable fit, consistent with a typical
SN~Ia luminosity and color. However, we consider this a ``plausible,''
rather than ``probable,'' SN~Ia, given the uncertain redshift and low
signal-to-noise of the light curve data.

%%%%%%%%%%%%%%%%%%%%%%%%%%%%%%%%%%%%
% TABLE: SN COUNTS AND UNCERTAINTY %
%%%%%%%%%%%%%%%%%%%%%%%%%%%%%%%%%%%%
%\begin{table*}{lccccc}
%\tablewidth{0pt}
%\tablecaption{\label{tab:sncount} SN~Ia counts}
%\tablehead{\colhead{environment} & \colhead{minimum} & \colhead{estimate} & \colhead{maximum} & \colhead{``Secure'' and ``Probable'' SNe~Ia} & \colhead{``Plausible'' SNe~Ia}}
%\startdata
%Cluster Early Type & 6 & 7   & 7  & D0, H5, K0, K18, R12, U4     & E12 \\  
%Cluster            & 7 & 8-9 & 9  & C1, D0, H5, K0, K18, R12, U4 & F12, E12\\
%Field $0.2<z<0.6$  & 0 & 0   & 0  & \nodata                      & \nodata \\
%Field $0.6<z<1.0$  & 3 & 5   & 6  & Z5, P9, H3                   & P1, G3, E12 \\
%Field $1.0<z<1.4$  & 5 & 5   & 7  & A4, C0, D6, G4, N33          & E12, N32 \\
%Field $z>1.4$      & 0 & 1   & 1  & \nodata                      & X26
%\enddata
%\end{table*}

