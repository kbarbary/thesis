
The 25 clusters are heterogeneous in the spatial distribution of their
member galaxies. Some clusters have very obvious cores with tens of
red elliptical galaxies, while in others no central overdensity is
obvious to the eye. For these later clusters, the exact ``center'' of
the cluster is nearly impossible to define from the ACS imaging
data. Because the luminosity analysis is not very dependent on having
an accurate center, we use a hybrid method to arrive at the cluster
positions listed in Table~\ref{tab:clusters}. For those clusters
having a clear central-dominant (cD) galaxy, the position of the cD
galaxy is used. Cluster Y has two close cD-like galaxies; the position
given is that of the Southwest galaxy, which is slightly brighter and
looks more centrally located with respect to other red elliptical
galaxies. For the 18 clusters lacking a cD galaxy we first attempt to
find a central overdensity of red-sequence galaxies by choosing the
position that maximizes the total luminosity of red sequence galaxies
within 0.1~Mpc (where a red-sequence galaxy is one that has a color
within 0.15~magnitudes of the cluster's red sequence and an
ellipticity less than 0.5). We then averaged the position of the red
sequence galaxies contained in that aperture (weighting by galaxy
luminosity) to get a slightly more refined (and unique) position. This
position is then evaluated by eye with respect to red elliptical
galaxies outside the 0.1~Mpc circle. If the position represents a
clear overdensity and seems consistent with galaxies outside the
circle, we use this position. This is done for clusters F (16
red-sequence galaxies within 0.1 Mpc), G (11), I (8), N (14), R (18),
T (19), W (16) and X (22). For clusters A (16), J (8), M (8) and S
(7), the algorithm gives a reasonable result, but the overdensity
seems offset from the larger distribution of galaxies and we chose a
(nearby) position by eye. For clusters H, K, L, O, P, U, the algorithm
either fails due to the presence of bright (lower redshift)
red-sequence interlopers in the field, or because there is no clear
overdensity of galaxies. For these clusters the center chose by eye is
less reliable.

