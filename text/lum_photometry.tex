
For galaxy detection and photometry we use the stacked $i_{775}$ and
$z_{850}$ band images of each cluster, which have total exposure times
in the range 1060 -- 4450~seconds and 5440 -- 16,935~seconds,
respectively. 

\subsubsection{Detection}

Galaxy catalogs are created using the two-pass ``Cold/Hot'' method of
running {\sc SExtractor} first developed by \citet{rix04a} and adapted
to this survey in \citet{meyers11a}.  In the Cold/Hot method, {\sc
SExtractor} is first run in dual-image mode using the $z_{850}$ image
for detection. The detection thresholds are generally set fairly high
(Table~\ref{tab:sextractor}) in order to detect only the brighter
galaxies. The ``segmentation map'', which represents which pixels
belong to a galaxy, is saved. {\sc SExtractor} is then run again (also
in dual-image mode using the $z_{850}$ image for detection) but with
more lenient detection parameters, in order to find the fainter
galaxies. This second catalog is then cleaned by throwing out galaxies
that fall on pixels belonging to galaxies in the first catalog. The
cleaned second catalog is then combined with the first catalog.  We
remove stars from the catalog based on the CLASS\_STAR and
FLUX\_RADIUS parameters from the $z_{850}$ image.

For consistency with \citet{meyers11a}, objects are detected using the
stacked $z_{850}$ images prior to background subtraction. However,
photometry for both $z_{850}$ and $z_{775}$ is done on the background
subtracted images. Thus, for the photometry image in dual-image mode,
the {\sc SExtractor} parameters BACK\_TYPE=MANUAL and BACK\_VALUE=0.0
are used.

\subsubsection{Photometry}

It is notoriously difficult to determine accurate total fluxes for
extended sources. MAGAUTO is generally thought to be {\sc
SExtractor}'s most reliable estimator of total flux, but can still be
wrong by a large amount depending on the galaxy.  MAG\_AUTO gives the
total magnitude of an object inside a flexible elliptical aperture,
with \emph{no aperture correction}. The size of the elliptical
aperture is based on the Kron radius. Although it varies from the
original definition, for the purposes of {\sc
SExtractor} \citet{bertin96a} define the Kron radius as
\begin{equation} \label{eq:kron}
R_1 = \frac{\Sigma R I(R)}{\Sigma I(R)}
\end{equation}
over the two-dimensional aperture. The elliptical aperture used has a
semi-minor axis length $KR_1$ where the user can chose the Kron factor
$K$. Theoretically, an aperture with, for example, $K=2$ will enclose
approximately 90\% of the light of a galaxy with a \citet{sersic68a}
profile, nearly independent of its S{\'e}rsic index of the
galaxy \citep{graham05a}.

On real images with noise, things are more complicated. Because the
outskirts of the galaxy are in the noise, {\sc SExtractor} will only
assign pixels in the inner part of a galaxy to each galaxy, and only
these pixels will be counted in determining $R_1$. As a result, $R_1$
is underestimated. The amount by which it is underestimated depends on
the S{\'e}rsic index of the galaxy and its surface brightness with
respect to the background noise. As a result, the fraction of light
that an actual MAG\_AUTO aperture encloses depends on the all above
factors. In order to use MAG\_AUTO magnitudes as an accurate estimate
of total light, we must find the aperture correction empirically for
our data. We do this using the Monte Carlo simulation described below.

In order to make the aperture correction as small as possible, we use
a relatively large Kron factor of 5.0 in our MAG\_AUTO
photometry. MAG\_AUTO is only used to determine $z_{850}$ magnitudes;
$i_{775}-z_{850}$ colors are determined using PSF matching and a
smaller aperture in \citet{meyers11a}.

