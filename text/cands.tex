
During the survey, our aim was to find as many supernovae as possible
and find them as early as possible in order to trigger spectroscopic
and NICMOS follow-up. Thus, software thresholds for flagging
candidates for consideration were set very low, and all possible
supernovae were carefully considered by a human screener. Over the
course of the survey, thresholds were changed and the roster of people
scanning the subtractions changed. As a result, the initial candidate
selection process was inclusive but heterogeneous, and depended
heavily on human selection. This made it difficult to calculate a
selection efficiency for the SN candidates selected during the
survey \citep[listed in Tables~3 and 4 of][]{dawson09a}. This is a
difficulty for a rate calculation since knowing the survey selection
efficiency is fundamental to calculating an accurate rate.

Therefore, in this chapter, we select an independent SN candidate sample
(without regard for the sample selected during the survey) using
automated selection wherever possible. Candidates are selected without
regard for cluster membership (which is only known from follow-up
spectroscopy once the candidate has already been found). We
determine SN types for both cluster and non-cluster SNe. The cluster
SNe are then considered further in Chapter~\ref{sec:clrate} in
calculating the cluster rate. The non-cluster SNe are considered
further in Chapter~\ref{sec:fieldrate} in calculating the volumetric
field rate.  The SN type determination here is also used to classify
candidates for use in the cosmological analysis from the
survey \citep{suzuki11a}.

The automated selection consists of initial detection in pairs of
subtracted images (\S\ref{sec:cands_search}; 86 candidates selected),
and subsequent requirements based on the light curve of each candidate
(\S\ref{sec:cands_lccuts}; 60 candidates remaining). The selection
efficiency for these two steps is later calculated via a Monte Carlo
simulation. In \S\ref{sec:cands_typing} we assign a type (SN~Ia,
core-collapse SN, or other) to each of the remaining 60 candidates
based on all data available (including triggered follow-up
observations). For this last step we do not calculate an efficiency or
completeness. Instead we estimate the classification uncertainty of
the assigned type for each candidate individually. For most candidates
the uncertainty in the type is negligible thanks to ample photometric
and spectroscopic data.
